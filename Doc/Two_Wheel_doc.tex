\documentclass[a4paper, 7px]{article}
\usepackage{amsmath}
\usepackage{graphicx}
\usepackage{hyperref}
\usepackage[a4paper,landscape, margin=2mm]{geometry}
\usepackage{tikz}
\usetikzlibrary{positioning}

\title{Two Wheel (name TBD)}
\author{Lorenzo Boglione}

\makeindex

\begin{document}
\maketitle
\newpage


\section{Structure}
\includegraphics[height=100mm]{Schematic_png}
\section{Kinematic}

\subsection{Variables}
Coordinates:
$$p = 
\begin{bmatrix}
	x\\ 
	y\\
	z\\ 
	\theta_x\\
	\theta_z\\ 
	
\end{bmatrix}
$$
Joint Variables:
$$q = 
\begin{bmatrix}
	\theta_R\\
	\theta_L\\
	d_L\\
	d_R\\
\end{bmatrix} =  
\begin{bmatrix}
	q_1\\ 
	q_2\\ 
	q_3\\
	q_4\\
\end{bmatrix}$$

Simplified system(q3=q4=const)

Generalized Coordinates:
$$u = 
\begin{bmatrix}
	d\\
	\theta_x\\
	\theta_z\\ 	
\end{bmatrix}$$

Joint Variables:
$$q = 
\begin{bmatrix}
	\theta_R\\
	\theta_L\\
\end{bmatrix} =  
\begin{bmatrix}
	q_1\\ 
	q_2\\ 
\end{bmatrix}$$
\subsection{Constraint Definition}
Diff drive constrain:
$$ \frac{dy}{dx}=tan(\theta_z)$$
$$ 
\begin{bmatrix} 
	sin(\theta_z)\\
 	cos(\theta_z)\\ 
 	0\\ 
 	0\\ 
 	0\\ 
\end{bmatrix} 
 * p = 0$$
We obtain the following G matrix
$$G = 
\begin{bmatrix}
c_{\theta_z} & 0 & 0 & 0\\
s_{\theta_z} & 0 & 0 & 0\\
0 & 1 & 0 & 0\\
0 & 0 & 1 & 0\\
0 & 0 & 0 & 1\\
\end{bmatrix}$$
$$\dot p = G \dot u = G 
\begin{bmatrix}
	v\\
	v_z\\
	\omega_x\\
	\omega_z\\
	
\end{bmatrix}$$
We can find a correlation between the u vector and q as
$$ \dot u = T \dot q$$
$$ T = 
\begin{bmatrix}
	\frac{r}{2} & \frac{r}{2} & 0 & 0\\
	0 & 0 & \frac{1}{2} & \frac{1}{2}\\
	0 & 0 & \frac{1}{d} & -\frac{1}{d}\\	
	-\frac{r}{\sqrt{d^2 + (d_R - d_L)^2}} & 				\frac{r}{\sqrt{d^2 + (d_R - d_L)^2}} & 0 & 0\\
	
\end{bmatrix}$$
$$G_q = G*T =  
\begin{bmatrix}
	r\frac{c_{\theta_z}}{2} & r\frac{c_{\theta_z}}{2} & 0 & 0\\
	r\frac{s_{\theta_z}}{2} & r\frac{s_{\theta_z}}{2} & 0 & 0\\
	0 & 0 & \frac{1}{2} & \frac{1}{2}\\
	0 & 0 & \frac{1}{d} & -\frac{1}{d}\\
	-\frac{r}{\sqrt{d^2 + (d_R - d_L)^2}} & 				\frac{r}{\sqrt{d^2 + (d_R - d_L)^2}} & 0 & 0\\
	 
\end{bmatrix}$$

$$\dot p = G_q \dot q$$


For the simplified system we already have the minimum number of coordinates so we only need to convert the generalized coordinates into physically meaning coordinate (joint variable)

$$ \dot q = T * \dot u$$
$$ \omega_R = \frac{v + \omega_z*\frac{d}{2}}{r} - \omega_y$$ 
$$ \omega_L = \frac{v - \omega_z*\frac{d}{2}}{r} - \omega_y$$
$$ T = 
\begin{bmatrix}
	\frac{1}{r} & -1 & \frac{d}{2 \, r}\\
	\frac{1}{r} & -1 & -\frac{d}{2 \, r}\\
\end{bmatrix}
$$
\newpage
\section{Dynamic}
For the dynamic we need to consider a third variable $\theta_y$

$$p = 
\begin{bmatrix}
	x\\
	y\\
	z\\
	\theta_x\\
	\theta_y\\
	\theta_z\\
\end{bmatrix}$$

And since no joint directly controls the orientation we can expand $G_q$ as follows:
$$G_q =  
\begin{bmatrix}
	r\frac{c_{\theta_z}}{2} & r\frac{c_{\theta_z}}{2} & 0 & 0\\
	r\frac{s_{\theta_z}}{2} & r\frac{s_{\theta_z}}{2} & 0 & 0\\
	0 & 0 & \frac{1}{2} & \frac{1}{2}\\
	0 & 0 & \frac{1}{d} & -\frac{1}{d}\\
	0 & 0 & 0 & 0\\
	-\frac{r}{\sqrt{d^2 + (d_R - d_L)^2}} & 				\frac{r}{\sqrt{d^2 + (d_R - d_L)^2}} & 0 & 0\\
\end{bmatrix}$$	

\subsection{Kinetic energy}
Let's take into consideratio one body at a time ignoring for the moment the motor's contribution: 
\begin{enumerate}
	\item Right Wheel
	\item Left Wheel
	\item Right Leg
	\item Left Leg
	\item Body
\end{enumerate}


\begin{huge}
	Change everything as function of u
\end{huge} \\


Express everything within respect to $\dot u = \begin{bmatrix}
v\\ v_z\\ \omega_x \\ \omega_y \\ \omega_z
\end{bmatrix} $

$$
K_1 =
\begin{cases}
	K_1 = \frac{1}{2} m_W v_{W_R}^2 + \frac{1}{2}\omega_{W_R}^T \Gamma_W \omega_{W_R}\\
	v_{W_R} = v + \omega_z \frac{\sqrt{d^2 + (d \, tan(\theta_x))^2}}{2}\\
	\omega_{W_R} = 
		\begin{bmatrix}
			\dot \theta_x\\
			\dot q1 + \dot \theta_y\\
			\dot \theta_z\\
		\end{bmatrix}
		=
		\begin{bmatrix}
			\omega_x\\
			\frac{v_{W_R}}{r} + \omega_y\\
			\omega_z\\
		\end{bmatrix}
		\\
		Gamma_{W_R} = ...
\end{cases}
$$

$$
K_2 =
\begin{cases}
	K_2 = \frac{1}{2} m_W v_{W_L}^2 + \frac{1}{2}\omega_{W_L}^T \Gamma_W \omega_{W_L}\\
	v_{W_L} = v  - \omega_z \frac{\sqrt{d^2 + (d \, tan(\theta_x))^2}}{2}\\
	\omega_{W_L} = 
		\begin{bmatrix}
			\omega_x\\
			\dot q2 + \omega_y\\
			\omega_z\\
		\end{bmatrix}=
		\begin{bmatrix}
			\omega_x\\
			\frac{v_{W_L}}{r} + \omega_y\\
			\omega_z\\
		\end{bmatrix}
\end{cases}
$$

$$
K_3 =
\begin{cases}
	K_3 = \frac{1}{2} m_L v_{L_R}^2 + \frac{1}{2}\omega_{L_R}^T \Gamma_L \omega_{L_R}\\
	v_{L_R} = v + \omega_z \frac{\sqrt{d^2 + (d \, tan(\theta_x))^2}}{2}\\
	\omega_{L_R} = 
		\begin{bmatrix}
			\omega_x\\
			\omega_y\\
			\omega_z\\
		\end{bmatrix}
\end{cases}
$$

$$
K_4 =
\begin{cases}
	K_4 = \frac{1}{2} m_L v_{L_L}^2 + \frac{1}{2}\omega_{L_L}^T \Gamma_L \omega_{L_L}\\
	v_{L_L} = v - \omega_z \frac{\sqrt{d^2 + (d \, tan(\theta_x))^2}}{2}\\
	\omega_{L_L} = 
		\begin{bmatrix}
			\omega_x\\
			\omega_y\\
			\omega_z\\
		\end{bmatrix}
\end{cases}
$$

$$K_5 = 
\begin{cases}
	K_4 = \frac{1}{2} m_B v_B^2 + \frac{1}{2}*\omega_B^T \Gamma_B \omega_B\\
	v_B = v + p_B X \omega_B\\
	\omega_B = 
		\begin{bmatrix}
			\omega_x\\
			\omega_y\\
			\omega_z\\
		\end{bmatrix}\\
		p_B = \begin{bmatrix}
		0\\
		0\\
		z\\
		\end{bmatrix}
		
\end{cases}
$$

\subsection{Potential energy}
With the same assumptio made before we can notice that we have only gravitational contribution.
We can consider the 0 plane at the wheel height and 

$$
U=
\begin{cases}
	U_1 = 0\\
	U_2 = 0\\
	U_3 = \frac{L_0  sin(\theta_x)sin(\theta_y)}{2}m_L\,g\\
	U_4 = \frac{L_0sin(\theta_x)sin(\theta_y)}{2}m_L\,g\\
	U_5 = zsin(\theta_x)sin(\theta_y)\,m_B\,g\\
\end{cases}
=
g(u)
$$

\subsection{External forces}

As first approximation we can neglect the dissipative forces as frictions.
The forces taken into account are the reaction force and the gravity force of the body mass.

$$f_r : 
\frac{\delta W_{f_r}}{\delta u} = 0
$$

$$f_g :
\begin{cases}
	
	\frac{\delta W_{f_g}}{\delta d} = 0\\
	
	\frac{\delta W_{f_g}}{\delta z} = 
\end{cases}$$
	
 


\subsection{Lagrange Equations}
Knowing both kinetic and potential energy we can start computing the left part of the lagrange equations

$$\frac{d}{dt}\bigg (\frac{\delta L}{\delta \dot u} \bigg) - \bigg( \frac{\delta L}{\delta u} \bigg )$$

$$\frac{\delta L}{\delta \dot u} = 
\begin{bmatrix}

	2 ( m_L + m_W + \frac{\Gamma_W}{r^2})v + \frac{ m_B ( 2 v - 2 \omega_y z)}{4 \sqrt{\omega_x^2 z^2 +  \omega_y^2 z^2 - 2 \omega_y v\, z + v^2+ v_z^2}} + \frac{2 \Gamma_W \omega_y}{r}\\
	
	\frac{m_B}{2 \sqrt{\omega_x^2 z^2 + \omega_y^2 z^2 - 2 \omega_y v \, z + v^2 + v_z^2}}v_z\\
	
	2(\frac{\Gamma_B}{2} + \Gamma_L + \Gamma_W)\omega_x + \frac{m_B}{2\sqrt{\omega_x^2 z^2 + \omega_y^2 z^2 - 2 \omega_y v \, z + v^2 + v_z^2}}\omega_x z^2\\	
	
	2(\frac{\Gamma_B}{2} + \Gamma_L + \Gamma_W) \omega_y + \frac{2 \Gamma_W}{r} v - \frac{m_B z (v - \omega_y z)}{2\sqrt{\omega_x^2 z^2 + \omega_y^2 z^2 - 2 \omega_y v \, z + v^2 + v_z^2}}\\

	2 \omega_z (\frac{\Gamma_B}{2} + \Gamma_L + \Gamma_W + \frac{d \, m_L (tan(\theta_x)^2 + d)}{4} + \frac{d \, m_W (tan(\theta_x)^2 + d)}{4} + \frac{\Gamma_W | d(tan(\theta_x)^2 + d) |}{4 r^2})
	
\end{bmatrix}
$$
We can make some assumpion, first the 

$$
\frac{d}{dt} \bigg( \frac{\delta L}{\delta \dot u} \bigg)= 
\begin{bmatrix}
	

 

\end{bmatrix}
$$


\section{Simplified Dynamic}
\subsection{Kinetic Energy}
$$
K1 = 
\begin{cases}
	K1 = \frac{1}{2} m_W v_1^2 + 1/2 \omega_1^T \Gamma_W \omega_1 \\
	v_1 = v + \frac{d}{2} \omega_z\\
	\omega_1 =
	\begin{bmatrix}
		0\\
		\frac{v1}{r}\\
		\omega_z\\
	\end{bmatrix}
\end{cases}
$$

$$
K1 = 
\begin{cases}
	K2 = \frac{1}{2} m_W v_2^2 + 1/2 \omega_2^T \Gamma_W \omega_2 \\
	v_2 = v - \frac{d}{2} \omega_z\\
	\omega_2 =
	\begin{bmatrix}
		0\\
		\frac{v2}{r}\\
		\omega_z\\
	\end{bmatrix}
\end{cases}
$$

$$
K3 = 
\begin{cases}
	K3 = \frac{1}{2} m_B v_3^2 + 1/2 \omega_3^T \Gamma_B \omega_3 \\
	v_3 = v + l \omega_y\\
	\omega_3 =
	\begin{bmatrix}
		0\\
		\omega_y\\
		\omega_z\\
	\end{bmatrix}
\end{cases}
$$
\subsection{Potential Energy}
$$ U = l \, cos(\theta_y) \, g$$

\subsection{Lagrange equation}
$$L = \frac{\Gamma _{b,\mathrm{yy}}\,{\left|\frac{\partial }{\partial t} \theta _{y}\left(t\right)\right|}^2}{2}+\frac{\Gamma _{b,\mathrm{zz}}\,{\left|\frac{\partial }{\partial t} \theta _{z}\left(t\right)\right|}^2}{2}+\Gamma _{w,\mathrm{zz}}\,{\left|\frac{\partial }{\partial t} \theta _{z}\left(t\right)\right|}^2+\frac{m_{w}\,{\left|-\frac{d\,\frac{\partial }{\partial t} \theta _{z}\left(t\right)}{2}+\frac{\partial }{\partial t} p\left(t\right)\right|}^2}{2}+\frac{m_{w}\,{\left|\frac{d\,\frac{\partial }{\partial t} \theta _{z}\left(t\right)}{2}+\frac{\partial }{\partial t} p\left(t\right)\right|}^2}{2}+\frac{m_{w}\,{\left|l\,\frac{\partial }{\partial t} \theta _{y}\left(t\right)+\frac{\partial }{\partial t} p\left(t\right)\right|}^2}{2}-g\,l\,\cos\left(\theta _{y}\left(t\right)\right)+\frac{\Gamma _{w,\mathrm{yy}}\,{\left|-\frac{d\,\frac{\partial }{\partial t} \theta _{z}\left(t\right)}{2}+\frac{\partial }{\partial t} p\left(t\right)\right|}^2}{r^2\,2}+\frac{\Gamma _{w,\mathrm{yy}}\,{\left|\frac{d\,\frac{\partial }{\partial t} \theta _{z}\left(t\right)}{2}+\frac{\partial }{\partial t} p\left(t\right)\right|}^2}{r^2\,2}$$




$$B \ddot{u} + g(u) = \gamma$$

$$B = 
\begin{bmatrix}
	\Gamma _{w,\mathrm{yy}}+3\,m_{w} & l\,m_{w} & 0\\
	l\,m_{w} & m_{w}\,l^2+\Gamma _{b,\mathrm{yy}} & 0\\
	0 & 0 & \Gamma _{b,\mathrm{zz}}+2\,\Gamma _{w,\mathrm{zz}}+\frac{d^2\,m_{w}}{2}+\frac{\Gamma _{w,\mathrm{yy}}\,d^2}{2\,r^2} \end{bmatrix}
$$
$$g(u) = 
\begin{bmatrix}
	0\\
	-g \, l \, \sin{\theta_y}\\
	0
\end{bmatrix}
$$
$$\gamma = \, forces \, on \, generalized \, coordinates = T^T \,  \tau$$
$$\tau = (T^+)^T \, \gamma$$
%\section{Controller}
%\begin{tikzpicture}
%	\node [draw, circle, minimum size = 0.6cm] (sum_P) at (0,0) {};
%	\node [draw, circle, minimum size = 0.6cm, below and right  of %sum_P]  (sum_I) {};
%\end{tikzpicture}
%\section{Simplified Controller}

\section{Controller}
\subsection{PD controller}
The controller is based on 2 main stages:
\begin{enumerate}
	\item Position/Orientation control
	\item Balancing control
\end{enumerate}

This division is necessary since we have 3 DoF but only 2 DoC (Given by the two wheel)

\subsubsection{Positon/Orientation control}
The first control stage consist in the PD control of the orientation and the position of the robot (considered as a point mass centered between the two contact point of the wheels).
While $\theta_z$ isn't influenced by the control of $\theta_y$,  and so can be controlled indipendently, the position and velocity is coupled.
We can find the following relation between $\ddot p$ and $ \theta_y$ 
$$\ddot p |_{\theta_y} = \frac{I_{p, \theta_y}}{I_{\theta_y}} * g * l sin(\theta_y)$$
This is obtained considering $\ddot \theta_y$ constant.
So the PD part of $p$ is converted into a corresponding $\theta_y$ that is further elaborated by the balancing stage.

\subsubsection{Balancing control}
The Balancing control stage purpose is to keep a given $\theta_y$ thats derived from the first stage



\section{ROS}

\subsection{Simulation}

The simulation is performed with Gazebo since it provides an easy interface with ROS.
The robot model along with the sensor's definition is done via an urdf file.
All the file needed for the simulation are in the two wheel sim package along with the urdf description

\subsection{Controller}
The interface between Gazebo and ROS is enstablished via the \href{https://github.com/ros-controls/ros2_control}{ros controls} 
and \href{https://github.com/ros-controls/gazebo_ros2_control}{gazebo ros2 control}.

\subsubsection{PID controller}
The controller is structured as 2 PID stages:
\begin{enumerate}
	\item Position/Velocity
	\item Inverted pendulum
\end{enumerate}

This separation is needed since the system despite having 3 generalized coordinates only 2 of them can be controller, in this case we only control $p$ and $\theta_z$

\paragraph{Position/Velocity controller}
The first stage controls positions and velocities in the space, so we only take into account $\theta_z$ and $p$.
$\theta_z$ provides directly an acceleration contribution while the result of $p$ controller produces the input for the next stage by computing the desired acceleration and transforming it into a desired $\theta_y$ thanks to the equation:
$$\ddot p = \frac{I_{p\theta_y}}{I_{\theta_y}}$$
Where $I_{p\theta_y}$ and $I_{\theta_y}$ are the (1,2) and (2,2) elements of B.


\paragraph{Inverted Pendulum controller}
The last stage purpose is to control the remaining degree of freedom $\theta_y$ this stage takes as input the desired $\theta_y$ and returns the needed acceleration.

\paragraph{Controller parameters}
The controller can be configured via a series of ROS parameters that are readed only when started.
\begin{itemize}
	\item body\textunderscore I\textunderscore diag: This is a vector of 3 elements containing the 3 main component of the body inertia, coupling component such as $I_{xy}$ are neglected.
	\item wheel\textunderscore I\textunderscore diag: same as the body but for the wheel.
	\item body\textunderscore mass: mass of the body.
    \item wheel\textunderscore mass: mass of the wheel.
    \item body\textunderscore height: distance between the wheel axle and the CoM of the body
    \item wheel\textunderscore radius: radius of the wheel.
    \item wheel\textunderscore distance: distance between the two wheel. 
    \item max\textunderscore angle: maximum $\theta_y$ allowed.
    \item $P$: vector of 6 elements containing the proportional feedback gain. Setting the first or the last 3 values to zero will create a velocity only controller. The 2 and 5 values MUST be nonnull in order to not fall.
    \item $I$: same as the previous but for integral, since the first 3 value are the integration of the last 3 it wouldn't make sense to set that values, so the last 3 values should be 0.
    \item $D$: same as the previous but for differential, again the last 3 are already the derivative of the first 3 so the first three items should be 0.
    \item tuning: boolean value that disables the first stage in order to enable a correct tuning of the IP stage.
    \item avg \textunderscore size: Since the robot doesn't perform well with abrupt $\theta_y$ changes is implemented a moving avarange filter on between the first stage and the IP stage. This slows the response but provide a more reliable control. This parameter specify the dimension of the moving avarange, the higher this value the smoother (and slower) will be the response.
   
\end{itemize}

The two wheel sim package already contains some values that works for a 100Hz controller.

\paragraph{Tuning}
For tuning purposes I added a topic (" /state ") that provides the current state of the robot, it can be useful integrated with the plot function of rqt while tuning the different PID values.

\subsubsection{MPC controller}
The MPC controller make use of an opensource library \href{https://github.com/nicolapiccinelli/libmpc}{libmpc} that depends on \href{https://eigen.tuxfamily.org/index.php?title=Main_Page}{Eigen3} \href{https://nlopt.readthedocs.io/en/latest/}{NLopt} \href{https://osqp.org/docs/}{OSQP}.
Since libmpc includes eigen3 as #include<Eigen/Core> but the apt package manager installs the library as <eigen3/Eigen/Core> a possible solution can be making a symbolic link with the following command\\
sudo ln -s /usr/include/eigen3/Eigen /usr/include/Eigen\\
sudo ln -s /usr/include/eigen3/unsupported /usr/include/unsupported\\
CAUTION: this can create conflict with other libraries or programs.

It also requires gcc 10.3 version

Since I can't make it work i'm moving to the control toolbox library 
\subsection{Odometry}
The controller also publish the odometry value for localization purposes. Since the IMU is already published by the plugin/node we only provide the one based on the wheel encoders, so only the position values and orientation around z are to be taken into account.

\subsection{Localization and Navigation}
The robot uses the \href{https://navigation.ros.org/index.html}{nav2} stack in order to obtain the robot position and plan the movement.

\subsubsection{Odometry}
In order to obtain a correct odometry the robot uses an extended kahlman filter that integrating the data provided by encoders and IMU it obtains a more accurate position of the robot. This node (part of the \href{https://github.com/cra-ros-pkg/robot_localization/tree/foxy-devel}{robot localization} package) also publish the odom->base_link transformation matrix to the /tf topic.

\subsubsection{SLAM}
A SLAM algorithm is used in order to create a map of the environment, the data is collected throught a lidar sensor on the front of the robot.

\end{document}
